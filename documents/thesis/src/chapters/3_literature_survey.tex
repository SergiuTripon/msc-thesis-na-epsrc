\chapter{\textsc{Literature Survey}}
\label{chapterlabel3}

The Background chapter introduced EPSRC and the problem addressed in this research project while also providing background to several concepts in network science, community detection algorithms and other forms of classification. This section presents the literature reviewed prior to commencing this project identify related works and determine whether studies with a similar focus as this one have already been completed.

\vspace*{1em}

\textbf{Community Structure on Wikipedia} \cite{wikipedia_community_structure} documents community structure thoroughly and provided a good starting point to the literature survey. The article describes the properties and application of community structures in networks. Moreover, it gives a detailed description of a significant number of community detection algorithms as well as several testing methods enabling the evaluation of the algorithms. The information on Wikipedia helped to gain a general understanding of the subject and identify the further direction of the survey.

\vspace*{1em}

\textbf{Community Detection and Mining in Social Media} \cite{community_detection_class} is a class taught by \textit{Lei Tang} (Yahoo! Labs) and \textit{Huan Liu} (Arizona State University) at Arizona State University. The class follows the teachings of a book with the same name \cite{tang2010community} published by \textit{Lei Tang} and \textit{Huan Liu} in 2010. The information from the class is available online in the form of PowerPoint presentations. Essentially, they provide a summary, highlighting the most important parts from the book. The class added the academic knowledge required to the literature survey while also expanding on the information gathered from Wikipedia in regards to community detection evaluation.

\vspace*{1em}

\textbf{Community structure in social and biological networks} \cite{girvan2002community} is a research paper authored by \textit{Michelle Girvan} and \textit{Mark EJ Newman} and published in 2002. The research paper introduces the community structure property, present in many networks. It defines the community structure of a network as network nodes linked together in densely connected clusters, between which there are only sparser connections.

Girvan and Newman also propose a method to detect communities in networks which is built based on the idea of centrality indices. The method is tested on both \textit{real-world} and \textit{computer-generated} networks whose community structure is both known and unknown. \textit{Real-world} networks used for testing include \textit{Zachary's Karate Club Study} and \textit{College Football}. The study used networks without a known community structure such as the \textit{Collaboration} and \textit{Food Web} networks. In the case of a known community structure, the authors found that the method performed well and identified the known community structure with high-sensitivity and reliability. Furthermore, in both cases of unknown community structure, the method still detected significant and informative community divisions.

This research paper was extremely beneficial as it provided further knowledge in the field of community detection as well as the concepts and methods required to complete this research project.

\vspace*{1em}

\textbf{Finding and evaluating community structure in networks}
\cite{newman2004finding} is another paper published in 2004 by \textit{Mark EJ Newman} and \textit{Michelle Girvan} and similar to their previous collaborative work, \textit{Community structure in social and biological networks}. In this publication, the authors propose a set of algorithms for identifying community structure in networks, which is defined as "natural divisions of network nodes into densely connected subgroups." Moreover, this paper represents the introduction of the community structure measure known as \textit{modularity}.

The proposed algorithms include shortest-path betweenness, resistor networks and random walks. A way to measure the strength of the community structure identified by the algorithms proposed is also introduced. The measure provides an objective metric for determining the number of communities a network should be divided into. Testing the algorithms on both \textit{computer-generated} and \textit{real-world networks}, \textit{Newman} and \textit{Girvan} demonstrate that the algorithms are extremely effective at identifying community structure.

The knowledge gained from this study was extremely helpful throughout the duration of this project. It explained how the algorithms proposed were built and how they work, while also providing examples of networks on which the algorithms were evaluated on. It also showed the progress of \textit{Newman} and \textit{Girvan} in the subject since their previous collaborative research effort.

\vspace*{1em}

\textbf{Topic oriented community detection of rating based social networks} is a study conducted by \textit{Reihanian et al.} \cite{reihanian2015topic} in 2015 focusing on community detection from the perspective of content analysis. Most community detection research chooses to focus only on the topological structure of the network. In a social network, for example, this is usually based on the number of communications among individuals. In contrast, this research paper aims to go further and explore and analyse the network's content flow.

The development process of the project commences by preprocessing and annotating topic labels and continues with the clustering of social objects and the creation of topic clusters. It concludes by applying a community detection algorithm to the produced topical clusters in order to identify the community structure within each cluster. Furthermore, a number of experiments are carried out on several data sets including \textit{Movielens 100k}, \textit{Book-Crossing}, \textit{CIAO}, \textit{MovieTweetings} and \textit{Movielens Latest}. It makes use of a performance metric, \textit{purity}, as defined by Zhao et al. \cite{zhao2012topic} which considers both topic and linkage structure. It identifies a maximum \textit{purity} value in each experiment as the topical clusters created in each data set incorporate members which are interested in the same unique topics.

Moreover, the study also compares the topic-oriented community detection proposed with the classical community detection method in which topical content is not analysed. It finds higher values of \textit{modularity} and \textit{purity} in the topic-oriented framework, as the basic network is partitioned into topical clusters, and members who have the same topic of interest are clustered into the same identified community.

\textit{Topic oriented community detection of rating based social networks} has similarities to this project in terms of the focus on topic analysis. It added a new, different perspective to the process of community structure detection which is extremely interesting and definitely a potential path of extending research.

\vspace*{1em}

\textbf{Community detection algorithms: a comparative analysis} is a research paper published by \textit{Lancichinetti et al.} \cite{lancichinetti2009community} in 2009 focusing on the comparison of a wide range of community detection algorithms. Two evaluation benchmarks are employed, the \textit{GN benchmark} by \textit{Girvan} and \textit{Newman} and the \textit{LFR benchmark} proposed by \textit{Lancichinetti et al.}

The community detection algorithms are tested on each evaluation benchmark and include the \textit{Fast greedy modularity optimization}, \textit{Exhaustive modularity optimization via simulated annealing}, \textit{Cfinder},  \textit{Markov Cluster}, \textit{Expectation-maximization} and \textit{Potts model approach}. Furthermore, a number of different graphs were used in the evaluation such as \textit{undirected} and \textit{unweighted} graphs, \textit{directed} and \textit{unweighted} graphs, \textit{undirected} and \textit{weighted} graphs and \textit{undirected} and \textit{unweighted} graphs with overlapping communities. On both evaluation benchmarks, the study found that the \textit{Dynamic algorithm (Infomap)} by \textit{Rosvall} and \textit{Bergstrom} performed the best. The \textit{Fast modularity optimization} by \textit{Blondel et al.} and the \textit{Potts model approach} by \textit{Ronhovde} and \textit{Nussinov} also had a good performance in the evaluation.

This comparative analysis served as a significant source of knowledge in terms of the community detection algorithms available, how they work, when they work best and on which networks. It definitely had an impact on the decisions made in this project in regards to community detection methods utilised. 

\vspace*{1em}

\textbf{On Accuracy of Community Structure Discovery Algorithms} is another comparative study authored by \textit{Orman et al.} \cite{orman2011accuracy} in 2011. It evaluates the majority of algorithms evaluated in \textit{Community detection algorithms: a comparative analysis}, with the exception of \textit{SpinGlass} by \textit{Reichardt} and \textit{Bornholdt} and \textit{Walktrap} by \textit{Pons} and \textit{Latapy}. A generated benchmark graph using the \textit{LFR benchmark} is used. This means that only artificial networks are taken into consideration while the community structure is already known. Each of the eleven community detection algorithms presented are tested on all generated network samples.

The study found that in all cases the \textit{Dynamic algorithm (Infomap)} by \textit{Rosvall} and \textit{Bergstrom} performed better than all other algorithms. \textit{Infomap} succeeded in identifying the communities even for high mixing coefficient values. Furthermore, \textit{Walktrap}, \textit{Markov Cluster}, \textit{Spinglass} and \textit{Louvain} also had an excellent performance level, although not as good as \textit{Infomap}. The research also discovered that for all algorithms, the higher the degree, the better the performance. Moreover, when the network size increases, some algorithms (\textit{Infomap}, \textit{Infomod}, \textit{Louvain}) performed better, others performed worse (\textit{Commfind}, \textit{SpinGlass}, \textit{LeadingEigenvector}, \textit{Radetal}) while the performance of the remaining algorithms (\textit{Walktrap}, \textit{FastGreedy}, \textit{MarkovCluster}) did not change.

This publication served as a reliability test which determined whether the findings in \textit{Community detection algorithms: a comparative analysis} were consistent when compared to other studies. The findings were indeed consistent as both studies identified more or less the same high-performance and low-performance algorithms. This helped to further solidify the decisions taken surrounding community detection algorithms in this thesis project.

\vspace*{1em}

\textbf{Analysis of Citation Networks} is a university project lead by \textit{Anita Valmarska} and \textit{Janez Dem\u{s}ar} at \textit{Jo\v{z}ef Stefan Institute} in \textit{Ljubljana}, \textit{Slovenia}. It focuses on the analysis of citation networks defined as directed networks where one research paper cites another. The data comprises of a collection of 63826 unique psychology-related papers crawled from \textit{Wikipedia} and \textit{Microsoft Academic Research data (MAS)}.

The resulting network consists of 3918 vertices connected by 5732 edges. The study employs the \textit{Louvain} method for the detection of community structure in the created network. The community detection algorithm detected 52 communities with the smallest cluster consisting of 7 research papers, while the largest cluster was constructed of 230 psychology-related publications. The study finds that the results produced have potential as the representation of the communities reveals sensible relationships between psychology sub-fields.

\textit{Analysis of Citation Networks} is the closest in terms of scale to this research project and provides an example of the data, algorithms and tools that other researchers used. Furthermore, it yields new knowledge and inspiration which contributed to the analysis process of this thesis project.