\chapter{\textsc{Introduction}}
\label{chapter:introduction}

Every day, humans produce valuable intelligence which can be used further in significant areas such as analysis, prediction, decision-making and many others. For example, when a user repeatedly watches TV series or films on Netflix, without realising, they provide Netflix with invaluable data in regards to their preferences including the genre, type and language. Their user account also provides useful information such as gender, age, when and how often they use Netflix, the device they access Netflix from and much more. This information can be analysed by Netflix as part of a research project in order to gather compelling insights from the data. The results of this research project can then have an impact on decision-making, for example, when deciding which series Netflix should add next to the platform based on user demand of a specific genre. Furthermore, it can also be used to improve the prediction of their recommender system. However, not everyone makes use of the gathered intelligence, and often, the opportunity to take advantage of it goes to waste.

One of the institutions that posses such important and substantial intelligence data is EPSRC. Currently, EPSRC holds a significant number of topics used by researchers to classify research grants when making a proposal. Furthermore, EPSRC are facing a difficult task in determining how finely or coarsely the topics should be defined. Subsequently, they are also uncertain on how this could be achieved, once the definition is determined.

This research project primarily seeks to cluster research topics into research areas using grant data collected from EPSRC. A secondary objective involves the clustering of researchers into researcher communities. The project aims to achieve this by applying a novel approach to a real world problem, involving graph theory, the concept of modularity, and community detection. Therefore, although the objectives are the clustering of topics and researchers, verifying whether this approach can be used to solve these kinds of problems is also a crucial objective. It is expected that this approach will provide a cogent way to group topics in terms of similarity and aid decision making regarding their definition.

A number of different interpretations of the data collected from EPSRC were turned into Networks of \textit{Topics} and \textit{Researchers}. A further two different interpretations of each network were considered. Also, three different edge weight interpretations and eight community detection algorithms were considered. These amounts were refined to one optimal combination of edge weight interpretation and community detection algorithm, while the two different interpretations of each network were compared. Further details regarding the networks, edge weight interpretations and community detection algorithms are provided later in the thesis.

It was found that using the \textit{Topic-grant} network produced a more rational topic clustering in comparison to the other interpretation of it, the \textit{Topic-researcher} network. In contrast, the \textit{Researcher-topic} network performed better than the \textit{Researcher-grant} network when considering the researcher clustering. Furthermore, the edge weight, \textit{weighted by the normalised number of grants} proved to be optimal. The results also showed that the \textit{Louvain} method produced the most rational and well-defined clustering compared to the other community detection algorithms considered.

\section{Structure of this Thesis}

The remainder of this thesis is structured as follows. Chapter 2 provides an introduction to EPSRC and the problem addressed as well as concepts used throughout the project. Chapter 3 reviews the literature consulted during the project and highlights its contribution. The methodology followed throughout the project is described in Chapter 4. The results of the project are presented, compared and discussed in Chapter 5 and their evaluation is also documented. In Chapter 6, the work carried out throughout the project is summarised and concluded, and further potential work is recommended.