\chapter{\textsc{Introduction}}
\label{chapterlabel1}

Every day, humans produce valuable intelligence which can be used further in significant areas such as analysis, prediction, decision-making and many others. For example, when a user repeatedly watches tv series or films on Netflix, without realising, they provide Netflix with invaluable data in regards to their preferences including genre, type and language. Their user account also provides useful information such as gender, age, when and how often they use Netflix, the platform they access Netflix from and much more. This information can be analysed by Netflix as part of a research project in order to gather compelling insights into the data. The results of this research project can then have an impact in decision-making. For example, when deciding which series Netflix should add next to the platform based on user demand of a specific genre. It can also be used to improve the prediction of their recommender system. However, not everyone makes use of the gathered intelligence, and often, the opportunity to take advantage of it, goes missing.

One of the institutions that posses such important and substantial intelligence data is EPSRC. Currently, EPSRC holds a significant number of topics used by researchers to classify research grants when making a proposal. Furthermore, EPSRC are facing a difficult task in determining how fine or coarse the topics should be defined. Subsequently, they are also uncertain on how this could be achieved, once the definition is determined.

This research project primarily seeks to cluster research topics into research areas using grant data collected from EPSRC. A secondary objective involves the clustering of researchers into research communities. The project aims to achieve this through applying a novel approach to a real world problem, involving graph theory, network science, the concept of modularity and community detection. Therefore, as much as the objectives are the clustering of topics and researchers, verifying whether this approach can be used to solve this kind of problems is also a crucial objective. It is expected that this approach will provide a cogent way to group topics in terms of similarity and aid decision making regarding their definition.

A number of different interpretations of the data collected from EPSRC were turned into \textit{Networks of Topics} and \textit{Researchers}. A further two different interpretations of each network were considered. Also, five different edge weight interpretations and eight community detection algorithms were considered. These amounts were refined to one optimal combination of edge weight interpretation and community detection algorithm, while the two different interpretations of each network were compared. Further details regarding the networks, edge weight interpretations and community detection algorithms are provided later in the thesis.

It was found that using the \textit{Topic network (Grants as edges)} produced a more rational topic clustering in comparison to the \textit{Researchers as edges} interpretation of it. In contrast, the \textit{Researcher network (Topics as edges)} network performed better than the \textit{Researcher network (Grants as edges)} when considering the researcher clustering. Furthermore, the edge weight normalized by the number of grants proved to be optimal. The results also showed that the Louvain method produced the most rational and well-defined clustering compared to the other community detection algorithms considered.

\section{Structure of this Thesis}

The remainder of this thesis is structured as follows. Chapter 2 provides an introduction to EPSRC and the problem addressed as well as concepts used throughout the project. Chapter 3 reviews the literature consulted during the project and highlights its contribution. The methodology followed throughout the project is described in Chapter 4. The results of the project are presented, compared and discussed in Chapter 5 and their evaluation is also documented. In Chapter 6, the work carried out throughout the project is summarised and concluded, and further potential work is recommended.

\iffalse
\section{Motivation and Scope}

\section{Scope}

Every day people produce some sort of useful and valuable intelligence, from booking a fight with easyJet to watching a movie on Netflix.

When using these services, people provide information regarding their preferences including their seat choice, or movie genre. After several uses, companies can make use of the intelligence gathered and improve the service experience through prediction. The intelligence can also be used in research.

However, a lot of the time, the intelligence goes unnoticed and the opportunity of making valuable use of it is missed.

This project aims to cluster research topics into research areas using current and historical grant data publicly provided by EPSRC.

EPSRC (Engineering and Physical Science Research Council) is the main UK government agency for funding research and training in engineering and the physical sciences \cite{epsrc_about_us}.

Recently during a talk, an EPSRC officer revealed the fact that there are a lot of complaints about the way research areas and researcher communities are defined within EPSRC. The officer stated that it was not clear how fine or coarse they should be divided, and that they should be constantly changed to reflect the funding and research trend.

EPSRC makes grant data available online through its Grants on the Web (GoW) service. Each grant record includes valuable information, from which various networks can be constructed and analysed including a topic and researcher-based network which this study focuses on.

This research project seeks to identify and analyse the community structure in the networks constructed using network analysis. Moreover, it aims to observe the data from a number of different perspectives. Firstly, the project analyses data from three different time periods: 1990-2000, 2000-2010 and current data. Secondly, the number and value of the research grants are incorporated as attributes to add a numerical and financial insight to the data. Correlations between the trends and community structure identified and other data factors may prove insightful.

I believe that this study will achieve its goals and produce valuable results. They will provide a deeper understanding of the community structure identified and how it can help decision-making in regards to the definition of topics and researcher communities within EPSRC. Furthermore, the time, numerical and financial elements will aid in providing answers and insights along the way.
\fi

\iffalse
\section{Research tasks}

This study has one specific area of research defined, the detection of community structure in the networks constructed from the collected grant data. Additionally, the study aims to analyse various other points of interest. Besides community detection, an extremely important breakthrough focuses on determining whether the community structures identified are reliable and their level of reliability. This also involves comparing them to other community structures identified in other projects with a similar theme.

Basic statistics of each network will also be computed and analysed  in order to gather insights into why the relationships within the networks are laid out the way they are. This will involve time comparison, as well as making use of other data attributes to help determine whether these attributes have an impact on the network's general or community structure. 

By combining the different areas of research identified, a number of research questions the project seeks to answer were compiled and are listed below:

\begin{enumerate}[itemsep=0cm]
    \item What are the significant network metrics of each network?
    \item Does each network hold a community structure?
    \item How large is the community structure identified in each network?
    \item Is there a similarity between the network community structure identified in this project compared to other research projects?
    \item Is the community structure identified in each network cogent?
    \item Do any interesting trends arise when comparing networks from different time periods?
    \item Does the number and value of research grants have an impact on the network's general structure and community structure?
\end{enumerate}
\fi

\iffalse
\section{Value}

Currently, the EPSRC Portfolio consists of 223 topics and 3175 grants are classified by one or more topics. The number of topics has increased by 15 since 2010 and 87 since 1990. Considering this growth, it means that defining research topics becomes increasingly difficult. As the number of topics enlarges, the number of complaints does too, as opinions on how topics should be defined differ significantly. In contrast, the analysis of grant data could be the answer to the problem, receiving a boost in value and potential.

This project aims to find an answer to this issue through the use of network analysis and community detection algorithms. At the moment, EPSRC provides a Visualising our Portfolio (VOP) service which allows users to visualise a number of networks based on research themes, areas, and organisations. However, VOP does not provide a visualisation of a topic-based network. Analysing a topic-based network automatically fills a gap which EPSRC doesn't cover and will potentially find compelling.

Initially, when a topic-based network is constructed, there is no division between topics, they are all treated the same. The only difference is their relationships to other topics and their attribute values. Let's assume that a task involves addressing a problem in regards to defining topics coarser or finer in a reliable way i.e. too many similar topics or not enough topics to cover the grants. Attempting to achieve the task by focusing on the full network will prove extremely difficult. Subsequently, a potential alternative could involve grouping similar topics manually, and then analysing each group separately. This would lower the difficulty level compared to the previous attempt, but will significantly increase the time it would take to complete the task.

However, if a technique employed computationally would be used to "slice" the network into a number of smaller clusters, the task would be complete in time and with more ease. This project uses network analysis and community detection methods optimal to the network in order to determine a its community structure. This brings various benefits and substantial value to the process of defining topics within EPSRC.

Firstly, the complainants will be pleased to hear that there is now a solution in place which addresses their grievances. This should subsequently have an impact on the number of future complaints received by the EPSRC. Secondly, EPSRC can take advantage of the results produced by this study to construct a topic and researcher-based network while also illustrating their community structure to accompany the other networks already built within the VOP service. Finally, as part of potential future work, the community detection algorithms used can be applied to the other networks in order to also detect their community structure and visualise it.
\fi

\iffalse
\section{Data, Algorithms and Tools}

EPSRC provides current and past data through its GoW service. Current data comprises of grant records categorised by research area and topic among other research entities, while past data consists of unclassified grant records. Each grant record holds various information including the researchers, topic classifications and value of the grant. However, it does not store information regarding the research area of a grant. This project only takes in consideration data from research grant and researcher records. Due to this lack of information, research area-based networks can only be constructed from current grant data. In contrast, the creation of research topic-based networks covering all time periods is possible. The Data section of the Methodology chapter provides further details on this matter.

Algorithms are used for two purposes within this project: to compute the network's statistics and detect its community structure. This study focuses on metrics such as the average degree, average weighted degree diameter, radius, density, modularity, average clustering coefficient and average path length. Moreover, a wide variety of community detection methods are considered including Infomap, Spinglass, Louvain, Label Propagation, Leading Eigenvector and Walktrap.

During this study, a number of tools are used for programming, analysis and visualisation purposes. Programming is done in Python and Bash using Jetbrains PyCharm IDE. Microsoft Excel is used for data validation. iGraph, NetworkX, Stanford SNAP aid the network analysis and the detection of community structure in the networks. The visualisation of the networks is created using Gephi, Graphistry and Adobe Photoshop.
\fi

\iffalse
\section{Results and Achievements}

The project set out a number of objectives and questions expected to be achieved or answered, at the end of this project. The achieved objectives represent the results and achievements of the project and are used to measure the project's success.

A number of community detection algorithms are applied to the networks and the underlying community structure is revealed. This provides understanding of how communities are formed from a funding perspective. It also offers a way to determine whether the current definition of research topics is too fine or too coarse and how to control it in the right direction.

Moreover, data spanning across 30 years is used to construct the networks, adding a significant temporal effect to analysis. It means networks from different time periods are compared and their differences explored. This will provide an indication of the impact that time has on the relationships and communities of networks.

Furthermore, number and value attributes are included in the data to allow the analysis of the network from a numerical and financial perspective. The number and value attributes represent the number of common entities and the value of the common entities, respectively. Correlations between the identified community structure and the attributes are investigated in order to bring further answers and insights to the analysis process.
\fi

\iffalse
\section{Self-assessment}
I believe that the outcomes of this study are both interesting and valuable. In particular, detecting community structure in networks was something new for me and led me to develop a growing interest in the subject.

Working on this project offered me the opportunity to find answers to real world issues through network analysis. The results of this study range from constructing a number of networks from external data to determining whether the networks hold a community structure and subsequently identifying it.

Three aspects that I found extremely interesting to analyse were the time, numerical and financial factors of the data. Comparing networks from different time periods provided interesting results that answer questions in regards to the transition of the network's general and community structure.

Additionally, observing networks from a numerical and financial standpoint led to insightful results providing answers to question regarding specific network behaviours.

Overall, gaining an understanding of how communities are structured within research networks was extremely engaging, as it answered questions in regards to how the definition of topics could be improved, which would remain unanswered without network analysis.
\fi