\chapter{\textsc{Conclusion}}
\label{chapter:conclusion}

In this project, graph theory was used as a novel approach to a real-world problem, involving the identification of topic and researcher clusters in publicly available data provided by EPSRC. The objective of the project was not only to provide a solution to the problem, but also to determine whether graph theory could provide the solution.

Furthermore, the problem that the project aims to solve was defined and extensive background information about EPSRC, the concept of modularity and community detection algorithms was provided. The state-of-the-art of the related topics was also reviewed. Additionally, the methods put into practice throughout every stage of the project were explained in detail.

The current and historical data collected from EPSRC was interpreted in a number of different ways and lead to several \textit{Topic} and \textit{Researcher} networks being constructed using both the current (2010 to 2016) and historical (1990 to 2000, 2000 to 2010) data sets. This was followed by an extensive comparison experiments on both current and historical data sets which aimed to identify an optimal edge weight and community detection algorithm that would result in a highly accurate and coherent clustering of topics and researchers. The candidates considered included three different interpretations of the edge weight attribute (\textit{unweighted}, \textit{weighted by normalised number of grants}, \textit{weighted by normalised value of grants}) and eight different community detection algorithms including \textit{Louvain}, \textit{Spinglass} and \textit{Fast Greedy}.

The comparative analysis resulted in a significant and valuable set of results. Firstly, edges \textit{weighted by the normalised number of grants} was determined as the optimal interpretation of the edge weight attribute due to its high modularity score and rational clustering produced. Secondly, the \textit{Louvain} method was "crowned" as the optimal community detection algorithm due to its high performance in the experiments and the well-defined nature of the community structure it identified. Finally, using the \textit{Topic-grant} and \textit{Researcher-topic} networks proved to be a better option than other network interpretations, as more cogent and balanced clusters of topics and researchers were produced.

This thesis, based on the knowledge available, represents the first approach of deploying graph theory in order to provide a solution to a real-world problem which at the moment, is specific to one organisation - EPSRC. With the data undergoing extensive experimental comparison, an evaluated solution featuring surprisingly high performance is identified and proposed.

In conclusion, this project represents clear evidence that the novel approach based on graph theory is of great value and, because it is not limited by any data set, it can be used to identify solutions to other real-world problems, perhaps the identification of topic communities within a network of newspaper articles.