\chapter{\textsc{Discussion}}
\label{chapter:discussion}

In this chapter, the key results produced are summarised, and the limitations and improvements of the project are outlined, while the improvements are translated into future work, which is recommended.

\section{Summary of key results}

At the end of this thesis project, three key results were produced. Firstly, a rational, well-defined and balanced clustering of current and historical research topics was achieved. Secondly, a concrete work collaboration network between researchers was discovered, however, its clustering resulted in a substantial amount of communities which slightly diminished the extent of its analysis and its potential value.

Finally, an optimal solution composed of an edge weight and a community detection algorithm was identified. The optimal solution identified outperformed all other solutions considered in the comparison experiments. Furthermore, the key results produced represent the fulfilment of the objectives set at the start of the project.

Additionally, the impact of the results depends on who makes use of which results and how those results are used. Specifically, the identification of rational topic and researcher clusters is valuable to EPSRC as it represents a coherent way of clustering topics based on similarity and relatedness, a solution to a problem which currently exists.

Furthermore, the data analysis performed and the network visualisations produced represent invaluable insights into both the current and historical data. On the other hand, the identification of an optimal combination of edge weight and community detection algorithm as a result of the exhaustive experiments conducted, is more important to someone who wants to address a different but similar clustering problem as the one addressed in this project.

\section{Limitations}

Due to the short amount of time available, the analysis performed on the \textit{Researcher} networks is slightly limited when compared to the analysis of the \textit{Topic} networks, in terms of the quantity of data collected and the extent of the analysis. However, this was justified in the end, as the value and importance of the results produced during the analysis of the \textit{Researcher} networks was minimised, when compared to the results of the \textit{Topic} networks analysis.

\section{Improvements}

The potential improvements of the project solely concern the analysis carried out on the \textit{Researcher} networks, as additional data can be collected and analysed. By collecting additional data about the researchers such as their department and organisation, would enable the discovery of potential correlations between the way researchers are clustered and their department, organisation or location.

\section{Future work}

Any research project, this included, can be extended to include more data or further experiments or analysis. In this case, four possible ways of extending the work carried out were identified:

\begin{itemize}[noitemsep]
    \item Extending comparison experiments stage to include more community detection algorithms
    \item Extending the analysis of the \textit{Researcher} networks
        \item Incorporate further data as node and edge attributes of the networks
    \item Comparison to a study carried out within a different context e.g. citation networks
\end{itemize}

\noindent Firstly, the comparison experiments stage which is partially concerned with the identification of an optimal community detection algorithm for the data, can be extended to include additional community detection algorithms to the ones already considered. This study carried out experiments using all community detection algorithms provided by the iGraph network analysis package. However, other packages are available, and consist of algorithms which are not included in iGraph. Therefore, these algorithms could be implemented and Incorporated into the experiments in order to determine whether better clustering results than the ones produced by the \textit{Louvain} community detection algorithm can be achieved.

Secondly, the analysis of the \textit{Researcher} networks was restricted by the amount of time available. By nature, the \textit{Researcher} networks are less informative than the \textit{Topic} networks as topics can be easily identified by people without additional knowledge, while researchers are cannot unless those same people know the researchers or work within EPSRC. However, the \textit{EPSRC GoW} service provides substantial public data, which is partly used in this project, but not fully. Both grant and researcher records consist of additional information which is not used in this project within fields such as the \textit{Department}, \textit{Organisation} and \textit{Industrial Sector Classifications} of a grant or the \textit{Organisation} and \textit{Department} of a researcher. Incorporating additional data in the form of node and edge attributes within both the \textit{Topic} and \textit{Researcher} networks will increase the level of contextual information available and provide a guaranteed extension of the analysis scope.

Finally, another way of extending this work can involve a comparative analysis of the results produced in this project and the results of a study with a similar end goal but which aimed to solve a different kind of practical or scientific problem. For example, this could represent a study involving the analysis of a citation network constructed based on the citations within a significant number of research papers. The objective would be to cluster research papers together in the hope that they share a common subject and researcher topic clusters are revealed. A comparative study may result in valuable insights into potential correlations between the two result sets while, also help cement graph theory as an optimal and valuable solution to clustering problems in both real-world and scientific scenarios.