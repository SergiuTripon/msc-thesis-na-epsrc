\chapter{\textsc{Conclusion}}
\label{chapterlabel7}

In this project, graph theory was used as a novel approach to a real-world problem, involving the identification of topic and researcher clusters in publicly available data provided by EPSRC. The objective of the project was not only to provide a solution to the problem, but also to determine whether graph theory could provide the solution.

Furthermore, the problem that the project aims to solve was defined and extensive background information surrounding EPSRC, the concept of modularity and community detection algorithms was provided. Works that contributed to the project were also described and their contribution outlined. Additionally, the methods put into practice throughout every stage of the project. were explained extensively.

The current and historical data collected from EPSRC was interpreted in a number of different ways and lead to several \textit{Topic} (\textit{Grants as edges}, \textit{Researchers as edges}) and \textit{Researcher} (\textit{Grants as edges}, \textit{Topics as edges}) networks being constructed using both the \textit{current} (2010 to 2016) and \textit{historical} (1990 to 2000 and 2000 to 2010) data sets. This was followed by an extensive stage of comparison experiments on both current and historical data sets which aimed to identify an optimal edge weight interpretation and community detection algorithm that would result in a highly accurate and coherent clustering of topics and researchers. The candidates considered included three different interpretations of the edge weight attribute (\textit{unweighted}, \textit{weighted by normalized number of grants}, \textit{weighted by normalized value of grants}) and eight different community detection algorithms including \textit{Louvain}, \textit{Spinglass} and \textit{Fast Greedy}.

The experimental comparison stage resulted in a significant and valuable set of results. Firstly, edges \textit{weighted by the normalized number of grants} was determined as the optimal interpretation of the edge weight attribute due to its high modularity score and rational clustering produced. Secondly, the \textit{Louvain} community detection algorithm developed by \textit{Blondel et al.}, was "crowned" as the optimal community detection method due to its high performance in the experiments and the well-defined nature of the community structure it identified. Finally, the \textit{Topic} (\textit{Grants as edges}) and \textit{Researcher} (\textit{Topics as edges}) networks proved to be better solutions compared to the other network interpretations as they produced more balanced and cogent clusters of topics and researchers.

This thesis, based on the knowledge available, represents the first approach of deploying graph theory in order to provide a solution to a real-world problem which at the moment, is specific to one organisation, EPSRC. With the data undergoing extensive experimental comparison, an evaluated solution featuring surprisingly high performance is identified and provided while its benefits and limitations are also highlighted. Due to the short amount of time available, the analysis performed on the researcher data is slightly limited in terms of the quantity of data collected and the amount of analysis carried out on it. However, this is justified in the end considering the lower value brought by the researcher data in comparison to the topic data. This limitation is discussed further in the the next section.

In conclusion, this project represents clear evidence that the novel approach based on graph theory is of great value and because it is not limited by any data set, it could be used to identify solutions to other real-world problems, perhaps the identification of topic communities within a network of newspaper articles.

\section{Further potential work}

Any research project, this included, can be extended to include more data or further experiments or analysis. In this case, four possible ways of extending the work carried out were identified:

\begin{itemize}[itemsep=0cm]
    \item Extending comparison experiments stage to include more community detection algorithms
    \item Extending the analysis of the \textit{Networks of Researchers}
        \item Incorporate further data as node and edge attributes of the networks
    \item Comparison to a study carried out within a different context e.g. citation networks
\end{itemize}

Firstly, the comparison experiments stage which is concerned with the identification of an optimal community detection algorithm for the data, can be extended to include additional community detection algorithms to the ones already tested. This study carried out experiments using all community detection algorithms provided by the iGraph network analysis package. However, other packages are available, and consist of algorithms which are not included in iGraph. Therefore, these algorithms could be implemented and added to the tests in order to determine whether better clustering results than the ones produced by the Louvain community detection algorithm can be achieved.

Secondly, the analysis of the \textit{Networks of Researchers} was restricted by the amount of time available. By nature, the \textit{Researcher networks} are less informative than the \textit{Topic networks} as topics can be easily identified by people while researchers are not unless those same people know the researchers personally or work within EPSRC.

However, the \textit{EPSRC Grants on the Web (GoW)} service provides substantial public data, which is partly used in this project, but not fully. Both researcher and grant records consist of additional information which is not used represented by fields such as the \textit{Department}, \textit{Organisation}, \textit{Industrial Sector Classifications} of a grant or the \textit{Organisation} and \textit{Department} of a researcher. Incorporating additional data in the form of node and edge attributes within both the \textit{Topic} and \textit{Researcher} networks will increase the level of contextual information available and provide a guaranteed extension of the analysis scope.

Finally, another way of extending this work can involve a comparative analysis of the results produced throughout this project and the results of a study with a similar end goal but which aimed to solve a different kind of practical or scientific problem. For example, this could represent a study involving the analysis of a citation network constructed based on the citations within a significant number of research papers. The objective would be to cluster research papers together in the hope that they share a common subject. A study revolving around such an analysis would provide valuable insights into whether there is a potential correlation between the two sets of results while it will also help cement the graph theory approach as an optimal and valuable solution to solving clustering problems within both real-world and scientific scenarios.