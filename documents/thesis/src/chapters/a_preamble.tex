\maketitle
%\makedeclaration

\newpage
\thispagestyle{empty}
\vspace*{1em}
\begin{center}
    \textbf{This side is purposely left blank.}
\end{center}

\begin{abstract} % 300 word limit
\thispagestyle{empty}

Collectively, humans take part in the everyday production of valuable data and intelligence with a significant use in areas including analysis, prediction and decision-making. The value of the data is primarily justified by the human judgement that contributed to its creation. Unfortunately, not everyone anticipates its value and the benefits it brings, and as a consequence, the opportunity of putting it to good use is often missed. Recently, EPSRC, an organisation which is in possession of substantial amounts of data, has been dealing with uncertainty with regards to defining research topics. Currently, it is unknown whether research topics should hold a more specific or broad definition. Additionally, once this is determined, how it could be achieved is also unknown. This models the problem that this research project aims to solve while also identifying an optimal solution in the process.

The primary objective of this research project is the application of a novel approach in graph theory to identify coherent clusters of topics within \textit{Networks of Topics} constructed using current (2010 to 2016) and historical (1990 to 2000, 2000 to 2010) data collected from EPSRC. A secondary objective involves the discovery of researcher clusters through the analysis of \textit{Researcher networks} using the same collected current and historical data.

A large-scale comparative analysis is carried out considering several network and edge weight interpretations and community detection algorithms with the aim of identifying an optimal solution which produces in the most well-defined, balanced, accurate and rational clustering of topics and researchers. The results show that the Louvain community detection algorithm applied on the \textit{Topic (Grants as edges)} and \textit{Researcher (Topics as edges)} networks using the normalised number of grants as the edge weight attribute resulted in the best topic and researcher clusters. This thesis proves that the novel approach to the problem is capable of making valuable use of the human judgement underlying the data.

\end{abstract}

\newpage
\thispagestyle{empty}
\vspace*{1em}
\begin{center}
    \textbf{This side is purposely left blank.}
\end{center}

\begin{acknowledgements}
\thispagestyle{empty}

First and foremost, I would like to express my sincere gratitude to my supervisor, Dr Shi Zhou, for his kind assistance and constructive comments, as well as his indispensable guidance on the early direction of this thesis project.

\vspace{5mm}

\noindent Also, I wish to thank University College London for giving me the opportunity to study at such a reputable institution and for equipping me with the knowledge required to thrive during this year of study and further in life.

\vspace{5mm}

\noindent Finally, I would like to thank my amazing parents and girlfriend, for their endless support and encouragement throughout my study.

\end{acknowledgements}

\setcounter{tocdepth}{2} 
% Setting this higher means you get contents entries for
%  more minor section headers.

\newpage
\thispagestyle{empty}
\vspace*{1em}
\begin{center}
    \textbf{This side is purposely left blank.}
\end{center}

\frontmatter

\renewcommand{\contentsname}{\textsc{Table of Contents}}
\renewcommand{\listtablename}{\textsc{List of Tables}}
\renewcommand{\listfigurename}{\textsc{List of Figures}}

\setcounter{tocdepth}{5}
\setcounter{secnumdepth}{5}
\tableofcontents
\listoffigures
\listoftables