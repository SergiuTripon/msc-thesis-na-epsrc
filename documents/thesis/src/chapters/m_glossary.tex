\chapter{\textsc{Glossary}}
\label{chapter:glossary}

This glossary explains how the network naming convention was formulated while also specifying the long form of the abbreviations used in the thesis.

\section{Naming convention}

Throughout this thesis project, networks are mentioned using a standard naming convention, as follows:

\begin{center}
\underline{term1} dash \underline{term2} network
\end{center}

\noindent where:

\begin{itemize}[noitemsep]
    \item term1 means that \underline{nodes} represent \underline{term1} in the network
    \item term2 means that \underline{edges} represent \underline{term2} in the network
\end{itemize}

\noindent The standard naming convention is used to name the 4 networks constructed in this project, as follows:

\begin{itemize}[noitemsep]
    \item \textbf{Topic-grant network}, in which:
    \begin{itemize}[noitemsep]
        \item \underline{first term (topic)} means that nodes represent \underline{topics}
        \item \underline{second term (grant)} means that edges represent \underline{grants}
    \end{itemize}
    \item \textbf{Topic-researcher network}, in which:
    \begin{itemize}
        \item \underline{first term (topic)} means that nodes represent \underline{topics}
        \item \underline{second term (researcher)} means that edges represent \underline{researchers}
    \end{itemize}
    \item \textbf{Researcher-grant network}, in which:
    \begin{itemize}
        \item \underline{first term (researcher)} means that nodes represent \underline{researchers}
        \item \underline{second term (grant)} means that edges represent \underline{grants}
    \end{itemize}
    \item \textbf{Researcher-topic network}, in which:
    \begin{itemize}
        \item \underline{first term (researcher)} means that nodes represent \underline{researchers}
        \item \underline{second term (topic)} means that edges represent \underline{topics}
    \end{itemize}
\end{itemize}

\clearpage

\section{List of abbreviations}

A number of abbreviations are also used in the thesis report, particularly in the comparison experiments, and their long form is the following:

\begin{itemize}[noitemsep]
    \item \textbf{EPSRC} stands for \underline{\textbf{E}}ngineering and \underline{\textbf{P}}hysical \underline{\textbf{S}}ciences \underline{\textbf{R}}esearch \underline{\textbf{C}}ouncil
    \item \textbf{GoW} stands for \underline{\textbf{G}}rants \underline{\textbf{o}}n the \underline{\textbf{W}}eb
    \vspace{1em}
    \item \textbf{uw} stands for edge \underline{\textbf{u}}n\underline{\textbf{w}}eighted
    \item \textbf{wnn} stands for edge \underline{\textbf{w}}eighted by \underline{\textbf{n}}ormalized \underline{\textbf{n}}umber of grants
    \item \textbf{wnv} stands for edge \underline{\textbf{w}}eighted by \underline{\textbf{n}}ormalized \underline{\textbf{v}}alue of grants
    \vspace{1em}
    \item \textbf{SG} stands for \underline{\textbf{S}}pin\underline{\textbf{g}}lass community detection algorithm
    \item \textbf{LV} stands for \underline{\textbf{L}}ou\underline{\textbf{v}}ain community detection algorithm
    \item \textbf{FG} stands for \underline{\textbf{F}}ast \underline{\textbf{G}}reedy community detection algorithm
\end{itemize}